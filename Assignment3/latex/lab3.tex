\documentclass[12pt]{article}
\renewcommand{\baselinestretch}{1.5}
\usepackage[utf8]{inputenc}

\usepackage{hyperref}
\hypersetup{linktoc=all}

\usepackage{amsmath}
\usepackage{amssymb}
\usepackage{listings}
\usepackage{algorithm2e}
\usepackage{graphicx}
\usepackage{float}

\begin{document}

\title{Parallel Processing Lab 3: \\ Introduction to CUDA - Matrix-Vector
Multiplication}
\author{Sarah Peachey \& Nathan Schomer}
\maketitle

\textbf{\textit{Abstract:}} GPU's can process SIMD code very quickly because
it can run many threads on different data with the same instructions. First
the CPU must create resources on the GPU and copy the data over, then the
kernel code is launched. Once the kernel is launched the CPU can either run
more code in parallel with the GPU or it can wait for the GPU kernel to
complete. Then the CPU would transfer the output from the GPU back to the CPU 
and then free the resources on the GPU. The kernel code is ran on every
thread in the thread blocks in the grid, which was allocated by CUDA
functions in the CPU code.   
\newpage

\vspace{-1.5cm}
\section{Naive Design}
\vspace{-0.25cm}

\qquad To calculate the matrix vector multiplication of $Ax=y$ where A is a
$n \times n$ matrix, x is a $n \times 1$ vector, and y is a $n \times 1$
vector. The naive approach is to create a thread to compute each element in
the y vector. That thread then loads all the elements of the x vector as
well as all the elements in that corresponding row of A. Then element-wise
multiply and sum those values and store it back into global memory in the
vector y. 

So two values are loaded from global memory for every two floating
point operations, so the arithmetic intensity of the algorithm is 1. Which
means the code is memory bound, because it only uses 10 of the available 8800
GFLOPs. To achieve the peak
performance rate 880 floating point operations must be computed per load
operation, since the GTX 1080 GPU has a peak processing rate of 8800 GFLOPs and
the memory bandwidth on the device is 320 GB/s. Another algorithm will be
applied in the next section that uses shared memory to increase the
arithmetic intensity.  

\vspace{1cm}
\begin{algorithm}[H]
	\KwData{Ad, Xd, Yd}
 	\KwResult{kernel to calculate the matrix vector multiplication}
 	initialization\;
	tid=threadIdx.y+(blockIdx.y*blockDim.y)\;
	yTemp=0\;
    \For{\texttt{i < MATRIX SIZE }}{
  	    yTemp+=A[tid*MATRIX SIZE+i]*X[i]\;
   	}
	y[tid]=yTemp\; 
   	
% \caption{Pthreads Numerical Integration}
\end{algorithm}

\pagebreak
\vspace{-0.6cm}
\section{Shared Memory Design}

\vspace{-0.4cm}
\qquad words 


\vspace{1cm}
\begin{algorithm}[H]
	\KwData{start\_x, end\_x, num\_traps, height}
 	\KwResult{estimated integral of function between start\_x and end\_x}
 	initialization\;
    \For{\texttt{i <  num\_threads}}{
  	    pack argument struct for thread i\;
   	}

    \For{\texttt{i < num\_threads}}{
  	    create thread i\;
        serially calculate sub-trapezoid i\;
   	}

    \For{\texttt{i < num\_threads}}{
        join each thread\;
        global\_sum = global\_sum + thread\_sum;
    }
\end{algorithm}


\pagebreak
\vspace{-0.6cm}
\section{Discussion of speed up}
\vspace{-0.4cm}
\qquad words 


\begin{equation}
    s = \frac{t_{serial}}{t_{parallel}}\label{eq1}
\end{equation}

\begin{table}[H]
\centering
\begin{tabular}{@{}|l|c|c|}
\hline
Thread Count & Global Speed-Up & Shared Speed-Up\\ \hline
2  &  &  \\ \hline 
4  &  &  \\ \hline 
8  &  &  \\ \hline 
16 &  &  \\ \hline 
\end{tabular}
\caption{Average Speed-Up over 10 Iterations}
\end{table}


\end{document}
