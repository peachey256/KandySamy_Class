\documentclass[12pt]{article}
\renewcommand{\baselinestretch}{1.5}
\usepackage[utf8]{inputenc}

\usepackage{hyperref}
\hypersetup{linktoc=all}

\usepackage{amsmath}
\usepackage{amssymb}
\usepackage{listings}
\usepackage{algorithm2e}
\usepackage{graphicx}
\usepackage{float}
\graphicspath{ {images/} }


\begin{document}

\title{Parallel Processing Lab 1: OpenMP}
\author{Sarah Peachey \& Nathan Schomer}
\maketitle

\textbf{\textit{Abstract:}} Write some words about openMP

\newpage

\tableofcontents

\newpage


\section{Histogram\label{histo}}
\qquad The histogram code generates an n length vector and then sorts all
the data into a histogram with 500 bins. In the multi-threaded aprroach each
thread is assigned a chunk of data to sort. This chunk of data is statically
assigned so they sort equal sized chunks. Each thread keeps a local copy of
the histogram containing the data that it sorted. Then after each thread
sorts its own chunk of data it reaches a critical section in which the
histogram in shared memory is incremented with the values in the local
histograms. As seen in the below table peak performance is reached with 8
threads, performance is not improved with 16 threads because of the overhead
associated with spawning more threads. 

\begin{center}
\hspace*{-2.5cm}
\begin{tabular}{@{}|c|c|c|c|}
\hline
Threads & 1 million items & 10 million items & 100 million items \\
\hline 
2 & 1.40 & 1.29 & 1.10 \\
\hline
4 & 2.33 & 2.35 & 2.44 \\
\hline
8 & 3.00 & 4.17 & 4.42 \\
\hline 
16 & 2.50 & 4.15 & 3.56 \\
\hline 
\end{tabular}
\hspace*{-2.5cm}
\end{center}


\section{Guassian Elimination\label{guass}}
\qquad words 

\end{document}
