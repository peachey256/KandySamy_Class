\documentclass[12pt]{article}
\renewcommand{\baselinestretch}{1.5}
\usepackage[utf8]{inputenc}

\usepackage{hyperref}
\hypersetup{linktoc=all}

\usepackage{amsmath}
\usepackage{amssymb}
\usepackage{listings}
\usepackage{algorithm2e}
\usepackage{graphicx}
\usepackage{float}
\graphicspath{ {images/} }

\begin{document}

\title{Parallel Processing Lab 1: OpenMP \\ Guassian Elimination}
\author{Sarah Peachey \& Nathan Schomer}
\maketitle

%\textbf{\textit{Abstract:}} Write some words about openMP

\newpage

\vspace{-1.5cm}
\section{Design}
\vspace{-0.25cm}
The Gaussian elimination function was initially parallelized
by adding OpenMP directives before the first set of for loops
and two of the nested for loops. However, these changes did not
yield any improvement in run time. Next, the algorithm was
implemented using a "ping pong" method as described by Dr. Kandasamy
during lecture. Using this method, the dependencies between the
elimination and division steps were removed since all calculations
are based on values stored in a read-only matrix. These changes
provided better performance over the original method. Pseudocode for
the "ping pong" method is shown on the next page.

\vspace{-0.6cm}
\section{Results}
\vspace{-0.4cm}
\qquad Algorithm run time is dependent on the number
of threads and current computer workload.
This is evident from the speed-up values shown 
in the table. Speed up was calculated using Equation 1. 
Values for "Auto" threads were determined
during the Eagles parade when the workload on the Xunil server were
much lower. Currently running processes were monitored using "top".
The remaining values were collected later when many more 
processes were running on Xunil.
The most speed up was achieved when OpenMP determined
the number of threads. Otherwise, the parallel version was generally
slower than the serial algorithm. As expected, the slowdown was
greater with a lower number of threads since the overhead incurred by
OpenMP was greater than the additional work throughput gained by
additional threads. 



\begin{equation}
    s = \frac{t_{serial}}{t_{parallel}}
\end{equation}

\pagebreak
\begin{center}
\hspace*{-2.5cm}
\begin{tabular}{@{}|c|c|c|c|c|}
\hline
Thread Count & 1024x1024 & 2048x2048 & 4096x4096 & 8192x8192 \\

\hline
Auto & 1.46 & 1.24 & 1.08 & 1.08 \\
\hline
2 & 0.31 & 0.27 & 0.29 & 0.31 \\
\hline
4 & 0.55 & 0.52 & 0.41 & 0.55 \\
\hline
8 & 0.63 & 0.89 & 0.87 & 0.89 \\
\hline 
16 & 1.20 & 0.89 & 0.85 & 0.90 \\
\hline
\end{tabular}
\hspace*{-2.5cm}
\end{center}

\vspace{1cm}
\begin{algorithm}[H]
\SetAlgoLined
\KwResult{Ping Pong Gaussian Elimination}
initialization\;

\For{each row in temp\_array}{
    OMP Directive\;
    \For{each element temp\_array}{
        \If{first row}{
            Perform Division Step
        } 

        \Else{
            Perform Elimination Step
        }
    }
    \If{Not last Iteration}{
        OMP Directive\;
        copy U\_array to temp\_array
    }
}
\end{algorithm}

%\pagebreak


\end{document}
