\documentclass[12pt]{article}
\renewcommand{\baselinestretch}{1.5}
\usepackage[utf8]{inputenc}

\usepackage{hyperref}
\hypersetup{linktoc=all}

\usepackage{amsmath}
\usepackage{amssymb}
\usepackage{listings}
\usepackage{algorithm2e}
\usepackage{graphicx}
\usepackage{float}

\begin{document}

\title{Parallel Processing Lab 2: \\ Pthreads Numerical Integration}
\author{Sarah Peachey \& Nathan Schomer}
\maketitle

\textbf{\textit{Abstract:}} pthreads is used to parallelize an executable function. 
When the pthread is created it is given a pointer to a function and a structure containing 
all the information and needed to run that function. The data structure than needs to be 
dereferenced, and the code can then run over a section of the data. 

\newpage

\vspace{-1.5cm}
\section{Design}
\vspace{-0.25cm}

\qquad The parallel numerical integration algorithm was designed such
that using $N$ threads and a trapezoid width of $W$, each thread would 
calculate $W/N$ sub-trapezoids assuming $W \mod N = 0$. With this design,
it would be expected to see a decrease in execution time with an
increased number of threads. 

When implemented, the algorithm was broken into 3 for loops which
follows the basic design pattern of a pthreads program. 
The first packed 
a custom struct with all necessary data for each thread (start x, end x, 
number of trapezoids, and height). The next loop created each thread,
passing in the respective argument struct and a pointer to the function
that calculates the current trapezoid. The last for loop, waits for
all threads to join and then adds the result from each thread to a 
global sum. The global sum is then returned to the calling function.
Pseudocode is included on the next page. 

\pagebreak
\vspace{1cm}
\begin{algorithm}[H]
	\KwData{start\_x, end\_x, num\_traps, height}
 	\KwResult{estimated integral of function between start\_x and end\_x}
 	initialization\;
    \For{\texttt{i <  num\_threads}}{
  	    pack argument struct for thread i\;
   	}

    \For{\texttt{i < num\_threads}}{
  	    create thread i\;
        serially calculate sub-trapezoid i\;
   	}

    \For{\texttt{i < num\_threads}}{
        join each thread\;
        global\_sum = global\_sum + thread\_sum;
    }
   	
\caption{Pthreads Numerical Integration}
\end{algorithm}

\pagebreak
\vspace{-0.6cm}
\section{Results}
\vspace{-0.4cm}
\qquad The pthreads numerical integration function was executed using 2, 4, 8, and 16 threads. Each thread count was executed 10 times and
the average speed-up was recorded. The results are listed in 
Table~1. Speed-up was calculated by dividing time of 
serial execution by the average execution time for each thread count
as shown in Equation 1.

As seen in Table~1, continuous speed-up was seen 
up to 8 threads. Then, there was a slight drop off with 16 threads.
Overall, the parallelized algorithm achieved substantial gains
in execution time over the serial version. 

\begin{equation}
    s = \frac{t_{serial}}{t_{parallel}}\label{eq1}
\end{equation}

\begin{table}[H]
\centering
\begin{tabular}{@{}|l|c|}
\hline
Thread Count & Speed-Up \\ \hline
2  & 4.70  \\ \hline 
4  & 13.91 \\ \hline 
8  & 15.50 \\ \hline 
16 & 14.77 \\ \hline 
\end{tabular}
\caption{Average Speed-Up over 10 Iterations}
\end{table}


\end{document}
