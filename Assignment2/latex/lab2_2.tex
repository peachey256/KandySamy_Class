\documentclass[12pt]{article}
\renewcommand{\baselinestretch}{1.5}
\usepackage[utf8]{inputenc}

\usepackage{hyperref}
\hypersetup{linktoc=all}

\usepackage{amsmath}
\usepackage{amssymb}
\usepackage{listings}
\usepackage{algorithm2e}
\usepackage{graphicx}
\usepackage{float}
\graphicspath{ {images/} }

\begin{document}

\title{Parallel Processing Lab 1: OpenMP \\ Guassian Elimination}
\author{Sarah Peachey \& Nathan Schomer}
\maketitle

\textbf{\textit{Abstract:}} pthreads is used to paralize an executable function. 
When the pthread is created it is given a pointer to a function and a structure containing 
all the information and needed to run that function. The data structure than needs to be 
dereferenced, and the code can then run over a section of the data. 

\newpage

\vspace{-1.5cm}
\section{Design}
\vspace{-0.25cm}
\qquad words 

\vspace{-0.6cm}
\section{Results}
\vspace{-0.4cm}
\qquad 


\begin{equation}
    s = \frac{t_{serial}}{t_{parallel}}
\end{equation}

\pagebreak
\begin{center}
\hspace*{-2.5cm}
\begin{tabular}{@{}|c|c|c|c|c|}
\hline
Thread Count & 1024x1024 & 2048x2048 & 4096x4096 \\
\hline
4 & 2.08 & 2.39 & 2.47 \\
\hline
8 & 2.70 & 4.03 & ()  \\
\hline 
16 & (2.15) & () & () \\
\hline 
32 & () & () & () \\
\hline
\end{tabular}
\hspace*{-2.5cm}
\end{center}

\vspace{1cm}
\begin{algorithm}[H]
\SetAlgoLined
\KwResult{Function that Performs Gaussian Reduciton}
initialization\;
Dereferance the struct; 
\For{each row in temp\_array}{
    \For{each element in section of row}
            Perform Division Step
	Barrier using a mutex
	Set principle diagonal to 1 
     \For{each row in section}
	 	\For{each item in row} 
            Perform Elimination Step
 	Barrier using a mutex 
	Make lower triangle 0 
	Barrier using mutex 
}
\end{algorithm}

%\pagebreak


\end{document}
