\documentclass[12pt]{article}
\renewcommand{\baselinestretch}{1.5}
\usepackage[utf8]{inputenc}

\usepackage{hyperref}
\hypersetup{linktoc=all}

\usepackage{amsmath}
\usepackage{amssymb}
\usepackage{listings}
\usepackage{algorithm2e}
\usepackage{graphicx}
\usepackage{float}

\begin{document}

\title{Parallel Processing Lab 1: Pthreads \\ Gaussian Elimination}
\author{Sarah Peachey \& Nathan Schomer}
\maketitle

\textbf{\textit{Abstract:}} pthreads is used to parallelize an executable function. 
When the pthread is created it is given a pointer to a function and a structure containing 
all the information and needed to run that function. The data structure than needs to be 
dereferenced, and the code can then run over a section of the data. 

\newpage

\vspace{-1.5cm}
\section{Design}
\vspace{-0.25cm}
\qquad The design, as seen in the pseudo-code on the following page, is
simply to spawn N threads with the use of a for loop. The loop points the
threads to a executable/function in which the gaussian reduction is
performed. To the perform the reduction data and information need to passed
in as well, but the built-in function that creates pthreads only excepts one
input. So if you need to pass in multiple pieces of data, a data structure
must declared and defined with all the input information. Then in the
function that the thread has to dereference the pointer to the input
structure and all attributes. So the function I made was called
$parallel\_gold$ and it accepted a TwoMat data structure called
$Matrices\_ptr$.
Originally the plan was to used a flip flop architecture so two matrices
were going to be passed it, so that's what I named it. As the code developed
a different design was used so the name seems inaccurate. But the TwoMat
data structure has a U, a, b, $num\_threads$, and tid properties. U is a
pointer to the elements in the input matrix U, a is the start of the section
that the thread will operate on, b is the end of the section that the thread
will operate on, $num\_threads$ is the number of threads (N), and tid is the thread
ID $[0, 1, ... N-1]$. The pthreads version is then designed very similar to
the serial design, except that the division is parallelized by each thread
getting a chuck on the row $[a, b]$ to operate on. Then a barrier is
implemented with a mutex lock. The elimination is parallelized by each
thread getting a chunk of rows $[a, b]$ to operate on. Then another barrier
using mutex lock, to make sure everything is synchronized before going to the
next k. One aspect of the parallel design that need to be cared for was when
k incremented, for some threads k would be greater then the lower bound (a)
of it's respective chunk, so if $(k+1)>a$ then $a++$. 
   	
\pagebreak
\vspace{-0.6cm}
\section{Results}
\vspace{-0.4cm}
\qquad As seen in the table on the next page, the best performance was seen
with 8 threads. At 16 threads the overhead for thread creation and
synchronization slowed down the program. The speed up time was calculated by
\eqref{eq1}. It is also interesting to note that for the same thread count
the speedup increased as the matrix got larger. This is most likely do to
the chunk size utilization. Since a updates for some threads as k
increments, that implies that the first thread is going to have a smaller
chunk to execute over time, and once k is larger then the upper bound (b) 
for that thread, it will no longer have anything to execute. So the larger
the chunks are, the more data, the longer each thread will continue to
evaluate it's respective chunk. A better design could have been to
recalculate a and b as k gets larger, so that each thread always has a chunk
of data to evaluate. 

\begin{equation}
    s = \frac{t_{serial}}{t_{parallel}}\label{eq1}
\end{equation}

\pagebreak

\begin{center}
\hspace*{-2.5cm}
\begin{tabular}{@{}|c|c|c|c|}
\hline
Thread Count & 1024x1024 & 2048x2048 & 4096x4096 \\
\hline
4 & 2.08 & 2.39 & 2.47 \\
\hline
8 & 2.70 & 4.03 & 4.76  \\
\hline 
16 & 2.16 & 3.53 & 4.62 \\
\hline
\end{tabular}
\hspace*{-2.5cm}
\end{center}

\vspace{1cm}
\begin{algorithm}[H]
\SetAlgoLined
\KwResult{Function that Performs Gaussian Reduction}
initialization\;
Dereferance the struct; 
\For{each row in temp\_array}{
    \For{each element in section of row}
            {Perform Division Step}
	Barrier using a mutex
	Set principle diagonal to 1 
     \For{each row in section}{
	 	\For{each item in row}{
            Perform Elimination Step}}
 	Barrier using a mutex 
	Make lower triangle 0 
	Barrier using mutex 
}
\end{algorithm}

%\pagebreak


\end{document}
