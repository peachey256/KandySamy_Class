\documentclass[12pt]{article}
\renewcommand{\baselinestretch}{1.5}
\usepackage[utf8]{inputenc}

\usepackage{hyperref}
\hypersetup{linktoc=all}

\usepackage{amsmath}
\usepackage{amssymb}
\usepackage{listings}
\usepackage{color}
\definecolor{light-gray}{gray}{0.9}
\lstset{numbers=right,
                tabsize=1, 
					 numberstyle=\tiny,
                breaklines=true,
                showtabs=false, 
					 backgroundcolor=\color{light-gray},
                numbersep=5pt,
                xleftmargin=.4in,
                xrightmargin=.4in}


\usepackage{algorithm2e}
\usepackage{graphicx}
\usepackage{float}

\begin{document}

\title{Parallel Processing Final: \\Jacobian Iterative Solver}
\author{Sarah Peachey \& Nathan Schomer}
\maketitle

\vspace{-1.5cm}
\section{Design}
\vspace{-0.25cm}

\qquad A while loop was used on the CPU to launch the GPU kernel and test
for convergence. The kernel took some shared memory that is the
size of the tile/thread block. Then loaded values from the A matrix and the
x vector into those shared memory tiles. The tiles ensured that the memory
accesses were coalesced, which increases performance. Each tile of shared
memory was then matrix vector multiplied and saved to a partial sum vector.
A reduction algorithm that also takes advantage of coalesced memory accessing 
was then performed on the partial sum vector to
calculate $\sum_{m\neq k}a_{m,k}x_m$. Which is then subtracted from the
corresponding value
in the b vector and is divided by the value on the diagonal. This is the new
x value which is then used to calculate the squared error.  

\newpage

\subsection{Showing the tiled approach matrix multiplication.} 
\lstinputlisting[language=C,
firstline=39, lastline=42, numbers=left]
{../jacobi_solver/jacobi_iteration_kernel.cu }

\subsection{Showing the reducation algorithm and difference error calculation.}
\lstinputlisting[language=C,
firstline=47, lastline=59, numbers=left]
{../jacobi_solver/jacobi_iteration_kernel.cu }


\pagebreak
\vspace{-0.6cm}
\section{Discussion of speed up}
\vspace{-0.4cm}

\qquad Speed-up was calculated for the jacobian iterative solver on the GPU
vs the CPU code with, Equation 1. 
As seen by the speed-up ratios in Table 1, with thread block the size of 32 threads there is
considerable speedup, and increased speedup as the size of the matrix
increased. This is probably do to the fact that since the matrix was larger
more computation had to be performed so the data transfer time had less impact on
the overal time. Furthermore, with a thread block the size of 16 threads the
speedup was mostly slower except for the matrix with 2048 elements which was
actually faster than with 32 sized thread blocks. But since most of the
speeds are within an allowable tolerance from 16 to 32 sized thread blocks,
I would say the code is not very sensitive to the size of the thread blocks. 

\begin{equation}
    s = \frac{t_{serial}}{t_{parallel}}\label{eq1}
\end{equation}

\begin{table}[H]
\centering
\begin{tabular}{@{}|l|l|c|}
\hline
Thread Block Size & Matrix Size & Speed-Up \\ \hline
16 & 512 & 2.131  \\ \hline
16 & 1024 &  9.539 \\ \hline
16 & 2048 &  28.503 \\ \hline
32 & 512 & 2.117  \\ \hline
32 & 1024 & 9.873  \\ \hline
32 & 2048 & 24.962  \\ \hline
\end{tabular}
\caption{Speed-Up calculated on Xunil-05}
\end{table}

\end{document}
